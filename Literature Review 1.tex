\documentclass[12pt, a4paper]{article}
\usepackage[a4paper]{geometry}
\usepackage{natbib}
\usepackage{url}
%\usepackage{setspace}
%opening
%\onehalfspace
\title{Literature Review - Assignment 1 \\Power in Movement by Sidney Tarrow, Ch 5}
\author{Brian Mc Donald - 52964702}

\begin{document}

\maketitle

\noindent The reading I chose for review assignment 1 is chapter 5 \emph{Political Opportunities and Constraints} and chapter 8 \emph{Mobilizing Structures and Contention} of \emph{Power in Movement} by Sidney Tarrow. To keep the exercise to a reasonable length Ive chosen to limit this review to chapter 5. The reason I chose this reading is that in keeping with the style of his introductory chapter Tarrow describes his ideas in a very engaging way by using historical cases to illustrate each concept. These chapters deal with two concepts central to the political process model and are key to answering the questions of why movements occur when they do and why are some movements successful while others fail? Its my interest in the importance of these questions which motivated me to choose this reading. 

Since the majority of examples he uses are of western European countries and of the United States an interesting analysis of the work is to judge its relevance in the wider context of movements outside of his geographical focus and also of examples which have occurred in the fifteen years since the book was published.
\section*{Chapter 5: Political Opportunities and Constraints}
This chapter begins with the explanation of why opportunities matter, an area that became a focus in the 1960 during a time of social upheaval an proliferation of social movements in the west. Prior to this focus on opportunities the two predominant theories used to explain the ``why" of mobilization of social movements were; in Europe the idea of the creation of ``life spaces" created by state welfare systems and in the United States by proponents of ``rational choice" theory. 

Both these theories found their limitations in their ability to explain why certain periods experienced more movements than others and why some countries experienced movements during certain periods while at the same time another comparably homogeneous states didn't. To illustrate this point he uses the examples of the United States and France in the 1930s which experienced widespread industrial action while Britain and Germany did not.

A comparative example from the past three years would be the comparing of the strength and success of social movement in Tunisia and Egypt during the ``Arab Spring" and comparing them with countries such as Bahrain which shares many contentious structural similarities.

Following up on the limitations of previous models Tarrow uses the case of the breakup of the Soviet Union as an example of the importance of opportunities. Up until the late 1980s opportunities for political contention and the operation of social movements the was very limited due to the centralized nature of control by the communist party of political institutions and its tools of repression. Drawing similarities between events in the Soviet Union and writings by Tocqueville where he writes that ``the most perilous moment for a bad government is one when it seeks to mend its ways".\citep{de1955old} In the soviet case this coincided with the introduction of \emph{perestroika}, a program of liberalizing reforms intended to allow the state to better adapt and compete with market-orientated influences. Tarrow goes onto describe a number of opportunities which arose following these reforms and which ultimately led to its demise.

In the  case of present-day China this opening of opportunities is seen by the government as an area of great relevance. Just this January it was reported in an editorial in the South China Morning Post how Wang Qishan, the new head of the Communist party's discipline watch dog recommended de Tocqueville's \emph{The Old Regime and the Revolution} as a required reading for the party's members.\citep{scmp} This signals the party's concern with the growing opportunities for political contention appearing in Chinese society.

\subsection*{Dimensions of opportunity}
The opportunities in which he writes of primarily consist of resources \emph{external} to movements. These opportunities may be both real or merely \emph{perceived}. They include both formal opportunities and informal and can be explained as four dimensions:
\begin{itemize}
\item \textbf{Increased access} This involved the opening up exploitable sections of the political system which can act as catalyst to furthering a movements influence. An example in the book being the breakup in the USA of the ``solid South" which afforded new opportunities to black electors.
\item \textbf{Shifting alignments} Once political alignments shift, instability may arise and give rise to rival blocs. This allows opportunities for these new blocs to outmanoeuvre previously dominant parties.  
\item \textbf{Divided elites} Conflict and splits in ruling elites can provide opportunities as they encourage political contention. The example given in the book was the split within the Soviet Communist Party which encouraged many satellite states to mobilize.
\item \textbf{Influential allies} These allies can become a source for resources that a movement would otherwise lack. This influence can take many forms, from a legal and decision making perspective from within the system to external influences such adding legitimacy to their cause.
\end{itemize}
\subsection*{Other aspects}
These four dimensions along with repression and facilitation constitute \emph{changes} in opportunities. Three other less dynamic aspects he then adds are; state strength; states strategies toward challenges and repression.  

State strength relates to the level of centralization of a country. More politically centralized state generally allow for greater power and by definition are more authoritarian. The greater power of such states is offset by their general lack of alternative institutional avenues in which allow for diffusing of political contention. Using the example of the United States and France in the 1960s he shows the course of student riots differed in the countries. In France while the protest took the form of riots, escalated quickly and was short lived, in the US the federalized structure of government caused the protests to be slower burning yet longer lived. Intersecting with state strength is the idea of having \emph{inclusive} or \emph{exclusive} strategies when dealing with the demands of political contenders. The use of these two aspects as a robust measure of opportunities or constraints I feel is limited. The complex and nuanced idea of both a ``strong/weak state" and ``prevailing state strategy" is not easily definable or reducible in terms of a dichotomy. 

He describes state repression of movements as taking two forms; depressing collective action or raising the cost of mobilization. As a long term responsive, raising the cost of mobilization is generally seen as the more effective option. In the case of depressive action, when collective action does eventually break out it tends to be an accelerated response due to the previous restrictions on actions and information. A modern examples of both of these forms would be the response to Iranian demonstrations in 2009 calling for liberal reforms. Initial depressive actions by the government such as executions of rioters and imprisonment of opposition politicians eventually moved to the raising of the costs of such demonstrations by imposing stricter media and protest laws.\citep{iran}

The choices of \emph{mode of repression} by governments may lead to what Tarrow calls \emph{repressive paradoxes} where by discouraging contentious politics by means of repression produces a radicalization of the opposition. A recent example of this would be the comparison between Syria who's suppression of collective action gave rise to an armed struggle with the country and Jordan which alternatively chose to allow greater political participation in the form of parliamentary elections. \citep{black}

Adding a further set of variables that contribute to contention, Tarrow explores the role of \emph{threats}; threats to interests; threats to values; and at times threats to survival. When dealing with resource mobilization those with the least to lose seem to be the most likely to participate in contentious politics, but the reverse can also be true where those with the most to lose are also those faced with the greatest threat due to inaction. The example given in the book shows the opportunities given to Yasser Arafat following public outrage at the expansion of Jewish settlements by Benjamin Netanyahu in 1997. Recently in Egypt, following  the rise to government of the Muslim Brotherhood and the subsequent rewriting of the constitution many groups such as the Coptic church, secularists and women's movements have mobilized. Much of this mobilization can be seen as a direct action due to the perceived threat faced by their inaction. \citep{egypt}

Tarrow introduces Tversky's concept of ``prospect theory" to explore the role of threat as a basis for contention. This theory explains that attitudes to risk depend of whether and actions outcomes are perceived as losses or gains, the former being of perceived as being of greater importance. 

\subsection*{Making, expanding and declining opportunities}
In acting on opportunities groups can often create further opportunities. These opportunities may be created for the group themselves or quite often for rival groups. Actions by one movement may also result in the creation of a counter-movement. 
Contentious action by movements can also create provide opportunities for elites. An example in Sri Lanka would be the sustained international pressure on its government regarding human rights abuses against Tamil rebels. This sustained pressure allowed the government to build stronger nationalist sentiment  within the country as a reaction to perceived attempted foreign intervention in domestic issues.\citep{sri}

The dynamic nature of opportunities means that while the expansion of opportunities is possible the reverse can also be true.  The presence of opportunities at one stage does in no way guarantee their presence into the future. 

\subsection*{Conclusion}
In proving dimensions in which opportunities for contention change Tarrow offers a robust explanation as to why movements occur. Less robust I feel are his explanations of the roles of threat, government strength and prevailing strategies. Government strength and prevailing strategies suffer from problems of ambiguity in what actually constitutes a strong government or in the case of prevailing strategies where government action may be considered reactionary instead of part of an easily definable overall strategy. In dealing with threats he highlight the weaknesses of the causal influence of threats by using the example of white participation in the US civil war. Other possible aspects which could be explored but would also suffer from similar problems would be examining the role of \emph{trust} and how the advancement and decline of networks of trust effect the question of why social movements occur. 
\\*
\\*
\\*
\\*
\bibliographystyle{custom}
\bibliography{lit3}
\end{document}
